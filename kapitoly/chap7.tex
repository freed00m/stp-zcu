\chapter{Markovské řetězce}
\section{Absorpční Markovské řetězce}
Markovský řetězec je absorpční má-li alespoň jeden absorpční stav a je-li přechod z každého neabsorpčního stavu do absorpčního stavu možný. Neabsorpční stavy nazveme \textbf{tranzientními}. (Jakmile se jednou dostaneme do absorpčního stavu, zůstaneme tam již napořád).

\begin{figure}
\begin{tikzpicture}[->,>=stealth',thick, node distance = 4cm]
\node[state](a){$A$};
\node[state, right of = a](b) {$B$};
\node[state, below right of = a, node distance = 2cm](c){$C$};

\path (a) edge node[above]{$0.5$} (b)
	  (a) [bend left] edge node[above]{$0.5$}(c)
	  (c) [bend left] edge node[below]{$0.25$}(a);
\path[min distance = 1cm] (b) [loop above] edge node[above]{$1$}(b)
						  (a) [loop above] edge node[above]{$0.25$}(a)
						  (c) [loop right] edge node[right]{$0.5$}(c);
\end{tikzpicture}

\caption{Absorpční a tranzientní Markovské řetězce}
\end{figure}

Markovský řetězec skončí v absorpčním stavu s pravděpodobností 1. (Pokud bude absorpčních stavů více, nevíme, ve kterém skončíme, ale bude mít pravděpodobnost 1). Odpovídající matice je

\[ \vec{P} =
\begin{bmatrix}
\frac{1}{4} & \frac{1}{4} & \frac{1}{2}\\
\frac{1}{2} & \frac{1}{2} & 0\\
0 & 0 & 1
\end{bmatrix}
\]

\subsection{Základní úlohy}
\begin{enumerate}[label=\arabic*)]
\item \textbf{pravděpodobnost, že Markovský řetězec skončí v daném absorpčním stavu}\br

	Nechť má Markovský řetězec $s$ stavů, z toho $r$ absorpčních a tímpádem $s-r$ je tranzientních. Máme k dispozici matici přechodu $\vec{P}$ je velikosti $s\times s$. Předpokládáme, že Markovský řetězec startuje z tranzientního stavu $a_i$ a ptáme se, s jakou pravděpodobností skončí v určitém absorpčním stavu $a$. Tuto pravděpodobnost označíme $d_i$. Pravděpodobnost, že Markovský řetězec přejde v jednom kroku ze stavu $a_i$ do stavu $a_j$ a potom přejde do stavu $a$, spočteme jako $P_{ij}\cdot d_j$ (ppst že MŘ přejde z $a_i$ do $a_j$ a potom do $a$). Úplná pravděpodobnost přechodu z $a_i$ do $a$ je

	\[ d_i=p_{i1}\cdot d_1+ p_{i2}\cdot d_2 + \ldots + p_{ij}\cdot d_s \]

	Maticový přepis této rovnice nám říká, že

	\[ \vec{d} = \vec{P}\vec{d},\quad \vec{d}=[d_1,d_2,\ldots,d_s] \]

	Tato soustava má více řešení. Vybíráme z nich jediný vektor $d$ tak, aby jeho prvek, odpovídající danému absorpčnímu stavu,  byl jedna a prvky odpovídající ostatním absorpčním stavům byly nulové. Pokud máme víc absorčních stavů, získaneme více vektorů $d$.

	\begin{note}{Příklad}
		Vektor $\vec{d}\t=[d_1,d_2,d_3,d_4,d_5]$, kde $d_1,d_5$ jsou absorpční stavy, $d_2,d_3,d_4$ jsou tranzientní stavy. Pravděpodobnost, že se dostanu do $d_1$ je
		\[ \vec{P} = [1,\cdot,\cdot,\cdot,0] \]

	\end{note}

\item \textbf{střední počet přechodů přes daný tranzientní stav}\br

	Nechť $t_{ij}$ označuje středí počet přechodů při startu z tranzientního stavu $a_i$ přes stav $a_j$ než se dostaneme do absorčního stavu.

	\[ t_{ij}=p_{i1}t_{1j}+p_{i2}t_{2j}+\ldots+p_{i,s-r}t_{s-r,j}+\delta_{ij} \]

	\[ \delta_{ij}=
	\begin{cases}
	1 & i=j\\
	0 & i\neq j; i,j = 1,\ldots, s-r
	\end{cases}
	\]

	(Důvod $\delta_{ij}$ je, že se výchozí stav započítává do počtu přechodů, pokud tam náhodou začínáme). Maticový zápis

	\[ \vec{T} = \vec{Q}\cdot\vec{T}+\vec{I} \]

	Matice přechodu $\vec{Q}$ vznikne z $\vec{P}$ vynecháním řádků a sloupců odpovídajícím absorpčním stavům, tj. $\vec{Q}$ je matice velikosti $(s-r) \times (s-r))$. Řešení je tedy

	\[ \vec{I}=(\vec{I}-\vec{Q})\cdot\vec{T}\Rightarrow \vec{T}=(\vec{I}-\vec{Q})^{-1} \]

\item \textbf{střední počet přechodů přes všechny tranzientní stavy před skončením ve stavu absorpčním}\br

	Využijeme znalosti z předchozí úlohy. $t_{ij}$ udává střední dobu (pokud se přechody uskutečňují ve stejných časových okamžicích) strávenou ve stavu $a_j$, pokud startujeme ze stavu $a_i$.

	\[ t_i=\sum_{j=1}^{s-r}t_{2j} \]

	Doba (střední počet průchodů) pobytu v tranzientních stavech před pohlcením v absorpčním stavu je

	\[ t = \vec{T}\cdot
	\begin{bmatrix}
	1\\1\\1\\1
	\end{bmatrix}
	\]
\end{enumerate}

\section{Bernoulliho řada}
Bernoulliho řada je množina nezávislých náhodných proměnných se stejným rozložením $[X(0),X(1),\ldots]$ a s hodnotami $\brs{0,1}$, kde $P(X(k)=1)=p$ a $P(X(k)=0)=q=1-p$. Tato řada je \textbf{stacionární} a platí

\[ \mean{X(k)}=p,\mean{X^2(k)}=p, \]
\[ \var{X(k)}=p-p^2=p(1-p)=p\cdot q  \]

Stejné pro Bernoulliho Náhodnou Veličinu, protože hustota pravděpodobnosti je stejná pro všechny časové okamžiky.

Pokud budou hodnoty $\brs{-1,1}$, jedná se o symetrickou Bernoulliho proces/řadu. Velice jednoduchý typ náhodného procesu, jedná se o přiřazení indexu náhodným proměnným, které jsou nezávislé a mají stejná rozložení. Složitější náhodné procesy lze generovat tak, že na vstup nějakého systému (filtr) aplikujeme tyto jednoduché procesy a na výstupu získáme korelované procesy.

\begin{figure}
\begin{tikzpicture}[->,>=stealth',thick,node distance = 2cm]
\node[input](in){};
\node[block, right of = in](filtr){filtr};
\node[output, right of = filtr](out){};

\path (in) edge node[below left]{nekorelovaný proces}(filtr)
	  (filtr) edge node[below right]{korelovaný proces}(out);
\end{tikzpicture}
\caption{Generování korelovaných procesů}
\end{figure}

\section{Bernoulliho náhodná procházka}
Nechť $W(k)$ je symetrický Bernoulliho process s hodnotami $\brs{-1,1}$ takový, že

\[ P(W(k)=1) = p = P(W(k)=-1)=q=\frac{1}{2} \]

Náhodná řada

\[ X(k) = \sum_{m=1}^k W(m) \]

a $X(0)=0$ a $W(0)=0$ je \textbf{Bernoulliho náhodná procházka}.

\begin{figure}
	\includegraphics[scale=0.3]{obrazky/random_walk.png}
	\caption{Akumulovaný majetek hráče po $k$-té hře, fenomén teplotního šumu}
\end{figure}

Součet nemá Bernoulliho rozložení a nabývá hodnot v oboru celých čísel \textbf{Z}). Definiční vztah lze přepsat

\[ X(k) =X(k-1)+W(k), \]

což je filtr s nekonečnou impulsní odezvou, s pólem $z=1$. Momenty Bernoulliovy řady $\mean{W(k)}=0,\mean{W^2(k)}=1$, momenty náhodné procházky

\begin{eqnarray*}
\mean{X(k)} & = & \mean{\sum W(m)} =  \sum \mean{W(m)} = \sum 0 = 0\\
\mean{X^2(k)} & = & \mean{\left(\sum_{m=1}^k W(m)\right)\cdot\left(\sum_{n=1}^k W(n)\right)}=\\
& = & \mean{\sum_m^k\sum_n^k W(m)W(n)}=\\
& = & \sum^{k,k}_{\substack{m,n\\m\neq n}}\underbrace{\mean{W(m)}}_0 \underbrace{\mean{W(n)}}_0+K = K
\end{eqnarray*}

Platí $\var{X(k)}= \mean{X^2(k)} - \mean{X(k)}^2 = \mean{X^2(k)} - 0^2 =  K$. Variance roste lineárně, proces není stacionární. Autokorelační funkce je ve tvaru

\begin{eqnarray*}
R_{XX}(k_1,k_2) &=& \mean{X(k_1)\cdot X(k_2)}=\mean{\left(\sum_{m=1}^{k_1}W(m)\right)\left(\sum_{n=1}^{k_2}W(n)\right)}=\text{momenty}=\\
& = & \min(k_1,k_2)
\end{eqnarray*}

Pokud to $W(k)$ je Bernoulliho řada bez symetrických hodnot s $ \brs{0,1} $ pak neříkáme Bernoulliho procházka, ale binomická sčítací řada.



\subsection{Poissonův sčítací proces}

Proces spojitý v čase, diskrétní

definujme časový interval $ \Delta t = \frac{t}{k} \;\;\;\; k \in T = R^k, k \in N$

a uvažujme náhodný proces $X(t) \in X(k \Delta t)$, kde $X(k \cdot 1)$ je bermouliho sčítací řada.

Taková definice procesu umožnuje zobecnění, kdy úspěch může nastav v libovolně dlouhém intervalu, ne pouze pro $ \Delta t = 1$

Uvažujme následující podmínky:
\begin{itemize}
	\item $X(k \Delta t)$ je binomická náhodná veličina s pavděpodobností úspěchu $p$ uměrnou délce intervalu $$ p =P (X(\Delta t) = 1) = \lambda \Delta t$$ kde $\lambda$ je parametr rychlosti.
	\item interval $\Delta t$ je dostatečně malý, aby v libovolném intervalu nastal úspěch pouze jednou. Pravděpodobnost výskytu úspěchu v jednom intervalu je roven $0$.
\end{itemize}

Potom $X(t)$ je Poissonův sčítací proces.


Jaká je ppst že v čase $t$ máme hodnotu $n$

\[ P(X(k \Delta t) = n) = (\lambda \Delta t)^n \cdot (1 - \lambda\Delta t)^{k-n} \]

Nechť $k \rightarrow \infty$, potom $\Delta t \rightarrow 0$ a tedy pravděpodobnost $P = \lambda \Delta t = 0$
ale střední hodnota musí být konečná.

\[ P \cdot k = \lambda \Delta t k = \lambda \frac{t}{k} k = \lambda t \]

aproximace

\[ P(X(t)=n) \approx \dfrac{(\lambda t)^n}{n!} e^{-\lambda t} \;\;\; n \in Z^+ \]

což je Poissonovo rozdělení s časově proměným parametrem $ \Delta t$

Definice Poissonův sčítací proces $X(t)$ je Náhodná Veličina diskrétní v úrovni, spojitý v čase s $X(0)=0$ takový, že ppst úspěchu v čase $t$ je dán Poissonovým rozdělením s parametrem $ \alpha = \lambda t$

pozn. na rozdíl od binomické sčítací čady může změna nastat v libovolném časovém okamžiku.

% \begin{figure}
% 	\includegraphics[scale=0.3]{obrazky/poisson_walk.png}
% \caption{Průběh poissonovi procházky}
% \end{figure}

Momenty:

\begin{itemize}
	\item $\mean{X(t)} = \lambda t$
	\item $\var{X(t)} = \lambda t$
\end{itemize}
Jde tedy o nestacionární proces

Příklady
\begin{itemize}
	\item přesný model radioaktivního rozpadu
	\item počet poruch na telefoní lince
	\item příchod zákazníků do obchodu
	\item Př. emaily jsou doručovány podle Poissonova procesu s rychlostí $\lambda = 0.2$ zpráv za hodinu. A emaily kontrolujete jednou za hodinu, jaká je ppst že nenajdete novou zprávu v inboxu? \[ P(1) = 0) = \dfrac{(-0.2 \cdot 1 )^0}{0!} e^{-0.2 \cdot 1} = 0.819 \] další \[ P(1) = 0) = \dfrac{(-0.2 \cdot 1 )^1}{1!} e^{-0.2 \cdot 1} = 0.164 \]
\end{itemize}

Zaměřme se na dobu mezi dvěma událostmi.

Poissonův proces je ISI (Indenpendent stacionary increments), můžeme tedy zkoumat libovolný časový interval

např. čas první události (označíme jí $\vec{Y}$ je to Náhodná veličina), pokud je čas první události větší než $t$, pak na intervalu $<0,t>$ je 0 událostí \[ P(Y>t) = P(X(t) = 0) = e^{-\lambda t} \;\;\;\; t >= 0 \]
pak \[ F_Y(t) = P(Y <= t ) = 1 - e^{-\lambda t} \]

což je  distribuční funkce s exponenciálním rozdělení \[ \mean{\vec{Y}} = \frac{1}{\lambda}, \;\;\; \var{\vec{Y}} = \frac{1}{\lambda^2} \]
jaká bude distribuční funkce času, než nastane $n$-tá událost (ozn. $\vec{Y_n}$)

Protože jsou intervali mezi událostmi nezávyslé, lze $\vec{Y_n}$ vyjádřit  součtem intervalů \[ \vec{Y_n} = \sum_{k=1}^n (\vec{Y_k} -\vec{Y_{k-1}}) \]

$\vec{Y_n}$ je tedy součet nezávislých náhodných veličin s exponenciálním rozdělením a má Erlangovo rozdělení \[ P_{\vec{Y_k}} (t) = \dfrac{\lambda^n t^{k-1}}{(n-1)!} e^{-\lambda t} \;\;\;\; t >= 0 \]

Momenty: \[ \mean{Y_n} = \frac{n}{\lambda},  \;\;\;\; \var{Y_n} = \frac{n}{\lambda}  \]

%V simulinku realizovat dirachovy pulsi s intervalem byl exponencialně rozložen, na výstupu integrátoru bude


\subsubsection{Autokovarianční (autokorelační) fce Poiss. processu}
\label{subs:Autokovarianční (autokorelační) fce Poiss. processu}

potřebujeme sdruženou hustotu ppsti pro dva časové okamžiky ($t_1$ a $t_2$), nečhť $t_2 > t_1$

\[ P(X(t_1) = n_1, X(t_2) = n_2) = P(X(t_2)=n_2 | X(t_1)=n_1 ) P(X(t_2)=n_1 ) \]


díky bezpamětovostí exponenciálního rozdělení

\[ P(X ) n+ x | X > n) P(X>x) \]

má podmíněná pravděpodobnost $P(X(t_2)=n_2| X(t_1)=n_1)$ Erlangovo rozdělení s parametry $\lambda (t_2 -t_1)$ a $r = n_2 - n_1$

Pokud by do okamžiku by do $t_1$ nastalo $n_1$ událostí, pak rozložení závisí na zbylém počtu $n_2-n_1$ v průběhu časového okamžiku $<t_1,t_2>$



\begin{align}
	P(X(t_1) = n_1, P(X(t_2)=n_2) &= \dfrac{\lambda (t_2-t_1)}{(n_2-n_1)} \cdot e^{-\lambda (t_2-t_1)} \cdot e^{-\lambda t_1} \\ &= \dfrac{1}{n_2} [\lambda(t_2-t_1)]^{n_2-n_1} (\lambda t_1)^{n_1}  {n_2 \choose n_1} e^{-\lambda t_2}
\end{align}


\section{Wienerův proces}
Wienerův proces je model fyzikálního jevu známého jako Braunův pohyb, kde se částice chaoticky pohybuje v kapalině díky náhodným srážkám s molekulami kapaliny. Wienerův proces lze odvodit z Bernoulliho řady, která nabývá hodnot $\brs{-\varepsilon,\varepsilon}$ pro malé $\varepsilon>0$. Předpokládejme, že se kolize odehrávají s rychlostí $\alpha$, tzn. odehrávají se v násobcích $\frac{1}{\alpha}$. Celkový počet kolizí za čas $t$ je $n=\lfloor\alpha\cdot t\rfloor$ (zaokrouhlování dolů na celá čísla).

\subsubsection*{Definujeme náhodný proces}
\[ X(t)=\sum_{m=1}^{\lfloor\alpha\cdot t\rfloor} W(m), \]

který je podobný Bernoulliho náhodné procházce, ale

\begin{enumerate}[label=\roman*)]
\item intervaly mezi událostmi jsou $\frac{1}{\alpha}$ namísto 1.
\item amplitudy jsou $\brs{-\varepsilon,\varepsilon}$ namísto $\brs{-1,1}$
\item horní mez je $\lfloor\alpha\cdot t\rfloor$ namísto $k$
\end{enumerate}

Wienerův proces se získá limitním přechodem $\varphi\to 0$ (spojitý náhodný proces) a $\frac{1}{\alpha}\to 0$ (proces spojitý v čase).

\[ Y(t) = \lim_{\substack{\varepsilon\to 0\\ \frac{1}{\alpha}\to 0}} X(t) \]

Jak budou vypadat momenty po limitním přechodu:

\[ \mean{Y(t)} = \lim_{\substack{\varepsilon\to 0\\ \frac{1}{\alpha}\to 0}} \mean{X(t)}=0 \]

Z toho plyne, že Wienerův proces má trvale nulovou střední hodnotu. V případě variace

\[ \mean{Y^2(t)} = \lim_{\substack{\varepsilon\to 0\\ \frac{1}{\alpha}\to 0}} \mean{X^2(t)}=\lim_{\substack{\varepsilon\to 0\\ \frac{1}{\alpha}\to 0}} \lfloor\alpha t\rfloor \varepsilon^2 \]

pro $\frac{1}{\alpha}\to 0, \lfloor\alpha t\rfloor=\alpha t$. Vztah upravíme vytknutím $t$ a tedy

\[ \mean{Y^2(t)} = t \lim_{\substack{\varepsilon\to 0\\ \frac{1}{\alpha}\to 0}} \frac{\varepsilon^2}{\frac{1}{\alpha}} \]

Aby měl proces smysl, musí $\mean{Y^2(t)}>0$ a zároveň $\mean{Y^2(t)}<\infty$. Proto musí $\frac{1}{\alpha}\to 0$ a $\varepsilon^2\to 0$ řádově stejně rychle.

\subsection{Definice Wienerova procesu}
Wienerův proces je náhodný proces spojitý v čase popsaný gaussovskou hustotou pravděpodobnosti (normální rozdělení), nebo:

\[ P_{x(t)}(x)=\frac{1}{\sqrt{2\pi\alpha\varepsilon^2 t}}e^{-\frac{x^2}{2\alpha\varepsilon^2 t}}, \]

která má nulovou střední hodnotu a varianci $\sigma^2_\lambda(t) = \alpha\varepsilon^2 t$. Počáteční podmínka Wienerova procesu je $P(X(t)=0)=1$ \uv{začínáme z nuly}. Je to proces spojitý v čase a v úrovni. Odpovídá výstupu integrátoru. Korelační a autokovarianční funkce jsou stejné z důvodu 0 středních hodnot, tedy

\begin{eqnarray*}
R_{XX}(t_1,t_2) &=&\mean{X(t_1)X(t_2)} = \mean{X(t_1)\cdot[X(t_1)+X(t_2)-X(t_1)]} = \\
& = & \mean{X^2(t_1)} + \mean{X(t_1)\cdot(\underbrace{X(t_2)-X(t_1)}_{\text{nezávislý přírůstek}})}= \\
& = & \mean{X^2(t)}+\underbrace{\mean{X(t_1)}}_0\cdot \underbrace{\mean{X(t_2)-X(t_1)}}_0 = \alpha\varepsilon t_1
\end{eqnarray*}

Předpokladem je $t_1<t_2$

Wienerův proces má stacionární nezávislé přírustky!

Často se označuje $\alpha\varepsilon^2 \triangleq \sigma^2$


\subsubsection{Gaussův process (normální)}
\label{subs:Gaussův process}

Náhodný proces $\brs{X_t,t\in T}$ se nazývá Gaussův, jestliže pro každé $t_1<t_2<\ldots<t_n\in T, n=1,2,\ldots$ platí

\[ p(\vecrow[x],\vecrow[t])=\frac{1}{\sqrt{(2\pi)^n |R_{XX}(t)|}} e^{-\frac{1}{2}(x-E[x])^T R_x^{-1}(t)(x-E[x])}, \]

kde $E[x]$ je vektor středních hodnot, $R_{XX}(t) = \mean{(X-\sigma)(X-\sigma)^T}$ je autokovarianční matice.

\[ R_{XX}(t)=E\brs{[x(t)\cdot E[x(t)]][\cdot]^T} \]


\section{Bílý šum}
\label{sec:Bílý šum}

Při modelování procesů v přírodě nebeo v technických processů se vždy po sestrojení modelu kontroluje zda jeho chování odpovídá realitě. Pokud se zjistí, že je model málo přesný se vylepší (často za cenu vyšší složitosti). Tato procedůra se často opakuje až do bodu, kdy další zesložitění nevylepší kvalitu modelu. To může být způsobeno dvěma faktory

\begin{itemize}
	\item v procesu zůstaly nepredikovatelné výkyvy (změny), pro které neexistují kauzální vztahy
	\item bylo dosaženo úrovně náhodných chyb v přístrojích použitých k měření.. na této úrovni lze opakovanými experimenty zjistit statistiky šumu
\end{itemize}

V mnoha případech je užitečným modelem šumu \uv{bilý šum}.

\subsubsection{Bílá náhodná řada}
\label{subs:Bílá náhodná řada}

\[ { X(n), n=1,2,...,\infty} \] je markovská řada, pro kterou platí, tzn. \[ P(X(k)|X(l)) = P(X(t)) \;\;\; k > l \] že $X(k)$ jsou vzájemně nezávyslé. Důsledkem je, že znalost $X(k)$ nám nepomůže pro predikování $X(l)$.

Pokud má $X(t)$ Gaussovksé rozložení, pak řada ${X(k)}$ je bílá Gaussovksá náhodná řada.

K její specifikaci pak stačí střední hodnota $\mean{X(t)} k \geq 1$ a kovarianční maticí $\mean{X(t) - \mean{X(t)}  (X(l) - \mean(X(l)) )} = C_{XX} (k,l)$

\[ C_{XX} (k,l) = Q_ \delta (k-l) \] kde $\delta(m)$ je kroneckrova $\delta$

\[ \delta(m) = \begin{cases} 1, & m = 0 \\ 0, & jinak \end{cases} \]

spojitá analogie bíle Gaussovské řady by se mohla hodit pro spojité systémy.

Uvažujmě stacionární G. proces s nulovou střední hodnotou a autokovarianční funkcí.

\[ R_{XX}^\rho (t + \tau, t) = R_{XX}^\rho (\tau) = \sigma^2 \dfrac{\rho}{2} e^{-\rho |\tau|} \]

pro velké $\rho$ má tento proces vlastnosti, jaké bychom pro bílý G. proces chtěli.

pozn. pokud $\rho$ bude celé číslo, ptoom řada.

\[ { \dfrac{R_{XX}^\rho(\tau)}{\sigma^2} , \rho_1 <\rho_2 <\rho_3 } \]

definuje Diracovo $\delta$

tedy \[ R_{XX}^\infty (\tau) = \sigma^2 \delta (\tau) \]

Formálně lze spojitý bílý G. proces $\{ X(t), t \;\; \in \tau \}$ definuje G. proces

\[ C_{XX}^\rho (t,\tau) = Q(t) \delta(t-\tau) \]

kde $Q(t)$ je pozitivně definitní kovarianční matice

Protože není Diracova funkce klasickou funkcí, je spojitý bílý G. proces \textbf{matematickou funkcí}.

bližší pohled

Jeho spektrální, hustotou je Fourierova transformace autokorelační funkce.

\[ S_{XX}^\rho (\omega) = \int_{-\infty}^{\infty} R_{XX}^\rho (\tau) e^{-j \omega \tau} d \tau  = \int_{-\infty}^{\infty} \sigma^2 \dfrac{\rho}{2} e^{-\rho |\tau|} e^{-j \omega \tau} d \tau = \dfrac{\sigma^2}{1+(\frac{\omega}{\rho})^2} \]


Pro $\rho \to \infty$

\[ S_{XX}^\rho (\omega) \to \sigma^2 \]

Vidíte že všechny frekvence jsou zastoupené stejně. Proto \uv{bílý}, páč barva o všech frekvencích stejně zastoupená je bílá.


Konstantní spektrální hustota vyžaduje nekonečnou energii.

$\rightarrow$ spojitý bílý šum není fyzikálně realizovatelný.

Bilý G. šum jako derivace W. procesu, ačkoliv W. proces není diferencovatelný v žádném slova smyslu.

bílý šum získáme jako $X(t) = Y'(t)$ kde $Y(t)$ je W. proces

autokorelační funkce W. process

\[ R_{YY} (t_1,t_2) = \sigma^2 (t_1,t_2) \min (t_2,t_1) = \dfrac{\partial^2}{\partial t_1 \partial t_2} R_{YY} (t_1, t_2) \]

pak

\[ \min (t_2,t_1) = t_1 pro t_1 < t_2 a t_2 pro t_1 > t_2  = \dfrac{\partial}{\partial t_1} \min (t_2,t_1) = \begin{cases} t_1 , & t_1 < t_2 \\ t_2, & t_1 > t_2 \end{cases} \]

což je jednotkový skok $1[t_2-t_1]$

\[ \to R_{XX} (t_1,t_2) = \sigma^2 \delta(t_2-t_1) \]

ačkoliv není bílý šum fyzikálně realizovatelný, v praxi se často objevu poces se spektrální hustotou konstantní pouze v určitém pásmu \uv{(už to není bílý šum)}


\[ S_{XX} (\omega) = \begin{cases}N, & |\omega| < B \\ 0, & jinak \end{cases} \]

\begin{center}
	\includegraphics[scale=0.2]{obrazky/autocorelation.jpg}
\end{center}
















