\chapter{Náhodné procesy}
\section{Náhodný proces}
\subsection{Definice}
\subsubsection{Hlavní definice}
Náhodným procesem budeme nazývat soubor $X$ náhodných proměnných $X=\brs{X_t:t\in T}$, kde množinu $T$ zpravidla chápeme jako nekonečnou množinu časových okamžiků. Pokud je množina $T$ konečná, pak se náhodný proces shoduje s náhodným vektorem. Je-li množina $T$ jednobodová, jedná se o náhodnou veličinu.

\subsubsection{Alternativní definice}
Nechť $(\Omega,\mathscr{F},P)$ je pravděpodobnostní prostor a $(\mathbb{R}^T,\mathscr{T})$ je topologický prostor všech zobrazení množiny $T$ do množiny reálných čísel $\mathbb{R}$ s přirozenou topologií $\mathscr{T}$. Pak zobrazení $X:\Omega\to\mathbb{R}^T$ nazveme náhodným procesem (s parametrem $t\in T$).

\begin{note}{Poznámka}
Náhodný proces $X$ lze chápat jako funkci proměnných $\omega\in\Omega$ a parametru $t\in T$, jejíž funkční hodnota $X(t,\omega)$ je pro pevné $t$ náhodná veličina a pro pevné $\omega\in\Omega$ je realizací náhodného procesu. 
\end{note}

\begin{note}{Poznámka}
Je-li množina $T$ diskrétní, tedy $T=\brs{t_1,t_2,\ldots}$, nazýváme jej procesem s diskrétním časem, nebo též \textbf{náhodnou posloupností}. Pokud $T=\mathbb{R}$, pak náhodný proces nazýváme procesem s časem spojitým.
\end{note}

Zobecněním náhodného procesu je vektorový náhodný proces, jehož definice se shoduje s definicí náhodného procesu, pouze místo množiny $\mathbb{R}$ se vyskytuje množina n-tic reálných čísel $\mathbb{R}^n$.

\subsubsection{Definice (alternativa)}
Mějme pravděpodobnostní prostor $(\Omega,\mathscr{F},P)$ a nechť $T\subset\mathbb{R}^1,T\neq\emptyset$ a zobrazení $X:\Omega\times T\to\mathbb{R}^1$ takové, že při každém pevném času $t_0\in T$ je $X(\cdot,t_0)$ náhodná veličina na $(\Omega,\mathscr{F},T)$, nazýváme náhodným procesem. Při pevně zvoleném $\omega_0\in\Omega$ nazýváme funkci $X(\omega_0,\cdot)$ \textbf{realizací náhodného procesu} $X$.

\begin{note}{Příklad}
Náhodná veličina $X_t$ popisuje počet impulsů registrovaných v telefonní ústředně za časový interval $\langle 0,t\rangle$. Pak $\brs{X_t}_{t\in T}$ je náhodný proces a $T=\langle0,\infty)$.
\end{note}

\begin{note}{Příklad}
$\brs{X_t}$ je fluktuační složka napětí na odporu v elektrickém obvodu v čase $t>0$. Fluktuace v důsledku náhodného pohybu elektronů (termický šum).
\end{note}

\begin{note}{Příklad}
Uvažujme populaci (lidí, zvířat, apod.), jejíž velikost kolísá náhodně vzhledem k náhodnosti narození, úmrtí a migrace. $\brs{X_t}$ určuje velikost populace v čase $t\in T$.
\end{note}

\begin{note}{Příklad}
Měříme teplotu v areálu ZČU a výstup zaznamenáváme termografem. Pak čára, kterou termograf nakreslí, je trajektorií náhodného procesu.
\end{note}

\section{Distribuční funkce náhodného procesu}
Analogicky k definici distribuční funkce náhodného vektoru lze očekávat, že distribuci nekonečné náhodné posloupnosti lze definovat vztahem

\[ F(X) = F(x_1,x_2,\ldots)=P(X_{t_1}(\omega)<x_1, X_{t_2}(\omega)<x_2,\ldots) \]

Avšak pravděpodobnost jevu vymezeného nekonečným systémem podmínek $X_{t_i}(\omega)<x_i,i=1,2,\ldots$ by mohla být nulová (prakticky je). To je nepoužitelné pro obecnou definici distribuční funkce náhodného procesu.\br

Ještě horší situace je u procesu spojitého v čase. Musíme se omezit na konečný počet podmínek. Korektním popisem náhodné posloupnosti je soubor všech konečně rozměrných distribučních funkcí.

\[ F(\vecrow[x],\vecrow[t])=P(X_{t_1}(\omega)<x_1,\ldots, X_{t_n}(\omega)<x_n), \]

kde $\brs{\vecrow[t]}$ je \textbf{libovolná} konečná podmnožina $T$.

\section{Hustota pravděpodobnosti náhodného procesu}
Hustota pravděpodobnosti náhodného procesu $\brs{X_t,t\in T}$ je definována jako derivace distribuční funkce.

\[ p(\vecrow[x],\vecrow[t])=\frac{\partial^n F(\svecrow[x],\svecrow[t])}{\vecrow[\partial x]}, \]

pokud derivace existuje. Přesněji

\[ F(\vecrow[x],\vecrow[t])=\int\limits_{-\infty}^{x_1}\cdots\int\limits_{-\infty}^{x_n}p(\vecrow[\xi],\vecrow[t])\vecrow[\d\xi] \]

\section{Stacionární náhodné procesy}
Stacionární náhodné procesy jsou takové procesy, jejichž stochastické vlastnosti jsou nezávislé na volbě počátečního okamžiku, resp. počátku časové osy. Pro libovolný časový okamžik $t_0\in T=\mathbb{R}$ musí platit

\[ F(\svecrow[x],\svecrow[t])=F(\svecrow[x],t_1+t_0,t_2+t_0,\ldots,t_n+t_0) \]

Například pokud by platilo $t_0=-t_1$, pak 

\[ F(\svecrow[x],\svecrow[t])=F(\svecrow[x],0,t_2-t_1,\ldots,t_n-t_1) \]

Distribuční funkce tedy nezávisí na $t_1$, ale pouze na rozdílech $t_2-t_1,t_3-t_1,\ldots,t_n-t_1$. Pokud například $n=1$, pak $F(x_1,t_1)$ nezávisí na čase, neboť platí 

\[ F(x_1,t_1)=F(x_1) \]

Podobně též pro $n=2$, kde

\[ F(x_1,x_2,t_1,t_2)=F(x_1,x_2,t_2-t_1) \]

Z toho plyne, že distribuční funkce stacionárního náhodného procesu je funkcí $2n-1$ proměnných. To samé platí i pro hustoty pravděpodobnosti.

\section{Ergodické náhodné procesy}
Pro zjišťování (měření) stochastických charakteristik náhodných procesů je důležitá vlastnost nazývaná ergodicita. Máme-li například střední hodnotu náhodného procesu, pak bychom podle definice měli změřit velké množství trajektorií a střední hodnotu odhadnout jako aritmetický průměr trajektorií v daném časovém okamžiku.\br

Z intuitivního hlediska se však nabízí možnost pozorovat pouze jednu z realizací a střední hodnotu odhadnout jako střední hodnotu funkce $X_t(\omega)$ danou výrazem

\[ \frac{1}{T}\int\limits_0^t X_t(\omega)\d t \]

Náhodné procesy, u nichž lze všechny charakteristiky zjistit tímto způsobem, se nazývají \textbf{ergodickými}. Ergodicitu často vykazují stacionární procesy, ale mohou vykazovat i procesy nestacionární.

\section{Markovovy (markovské) náhodné procesy}
Nechť je dán náhodný proces $\brs{X_t,t\in T}$ systémem všech konečně rozměrných hustot pravděpodobnosti.

\[ p(x_n,x_{n-1},\ldots, x_1,t_n,t_{n-1},t_1), \]

kde $t_1<t_2<\ldots<t_n$ je uspořádaná množina časových okamžiků z $T$. Předchozí hustotu pravděpodobnosti lze vyjádřit

\[ p(x_n,\ldots,x_1,t_n,\ldots,t_1) = p(x_n|x_{n-1},\ldots,x_1;t_n,\ldots,t_1)\cdot p(n_{n-1},\ldots,x_1,t_{n-1},\ldots,t_1) \]

Hustotu $p(n_{n-1},\ldots,x_1,t_{n-1},\ldots,t_1)$ lze vyjádřit 

\[ p(x_{n-1},\ldots,x_1,t_{n-1},\ldots,t_1) = p(x_{n-1}|x_{n-2},\ldots,x_1;t_{n-1},\ldots,t_1)\cdot p(n_{n-2},\ldots,x_1,t_{n-2},\ldots,t_1) \]

Jestliže pro každou z podmíněných hustot platí

\[ p(x_k|x_{k-1},\ldots,x_1,t_k,\ldots,t_1)=p(x_k|x_{k-1};t_k,t_{k-1}) \]

pro $k=2,3,\ldots,n$. Pak náhodný proces $\brs{X_t,t\in T}$ nazveme \textbf{Markovovým náhodným procesem}. Hustota pravděpodobnosti pro Markovovův náhodný proces

\[ p(x_n,x_{n-1},\ldots,x_1,t_n,\ldots,t_1)=\prod_{k=1}^np(x_k|x_{k-1};t_k,t_{k-1}), \]

kde $p(x_1|x_0;t_1,t_0)=p(x_1,t_1)$. Rozložení náhodné proměnné $X_{t_k}$ v čase $t_k$ Markovova procesu $\brs{x_t,t\in T}$ při známé hodnotě $X_{t_{k-1}}(\omega)$ nezávisí na průběhu před časovým okamžikem $t_{k-1}$. Tuto vlastnost má proces tvořený vektorem stavu stochastického systému.

\section{Gaussův proces (normální)}
Náhodný proces $\brs{X_t,t\in T}$ se nazývá Gaussův, jestliže pro každé $t_1<t_2<\ldots<t_n\in T, n=1,2,\ldots$ platí

\[ p(\vecrow[x],\vecrow[t])=\frac{1}{\sqrt{(2\pi)^n |R_x(t)|}} e^{-\frac{1}{2}(x-E[x])^T R_x^{-1}(t)(x-E[x])}, \]

kde $E[x]$ je vektor středních hodnot, $R_x(t)$ je kovarianční matice

\[ R_x(t)=E\brs{[x(t)\cdot E[x(t)]][\cdot]^T} \]

\section{Funkce náhodného procesu}
\subsection{Autokorelační funkce}
Autokorelační funkce náhodného procesu $X(t)$ je 

\[ R_{xx}(t_1,t_2)=E[X(t_1)X(t_2)] \]

Obecně se jedná o funkci dvou časových okamžiků.

\subsection{Nekorelovaný proces}
Náhodný proces se nazývá nekorelovaný, pokud

\[
R_{xx}(t_1,t_2) = 
\begin{cases}
	E[X^2(t_1)], & t_1=t_2\\ 
	E[X(t_1)]E[X(t_2)],&t_1\neq t_2
\end{cases}
\]

\subsection{Autokovarianční funkce}
Autokovarianční funkce náhodného procesu $X(t)$ je

\[ C_{xx}(t_1,t_2) = E[(X(t_1)-E[X(t_1)])\cdot (X(t_2)-E[X(t_2)])] \]

\begin{note}{Poznámka}
Platí, že autokovarianční funkce je rovna rozdílu nekorelovaného procesu a středních hodnot náhodných procesů, tj.

\[ C_{xx}(t_1,t_2) = R_{xx}(t_1,t_2) - E[X(t_1)]E[X(t_2)] \]
\end{note}

Pro nekorelovaný proces platí

\[
C_{xx}(t_1,t_2) = 
\begin{cases}
	C_{xx}(t_1,t_1)=E[(X(t_1)-E[X(t_1)])^2]\\
	0; t_1\neq t_2
\end{cases}
\]

\section{Stacionarita v širším smyslu}
Náhodný proces je stacionární v širším smyslu, pokud jeho střední hodnota je konstantní a autokovarianční funkce závisí pouze na rozdílu časových okamžiků a ne na jejich konkrétní hodnotě.

\[ E[X(t)] = m_x\forall t\in T \]

\[ C_{xx}(t_1,t_2)=C_{xx}(t_2-t_1),\forall t_1,t_2\in T \]

Často je značeno $C_{xx}(t_2-t_1)=C_{xx}(\tau)$. Pro nekorelvané procesy, které jsou stacionární v širším smyslu, platí

\[ C_{xx}(\tau) = \sigma_x^2\cdot\delta(\tau) \]

Pokud je proces spojitý v čase, $\delta$ představuje Diracův impuls, v diskrétním případě se jedná o Kroneckerovo deltu.

\begin{figure}
\begin{tikzpicture}[>=stealth', thick]
\draw[->] (0,0) -- node[below, near end]{$x$}(6,0);
\draw[->] (2,-1) -- node[left, near end]{$y$}(2,2);
\draw[->, blue] (2,0) -- node[right]{$\delta$}(2,1.5);
\end{tikzpicture}
\caption{Diracův impuls a Kroneckerovo delta}
\end{figure}

\begin{note}{Poznámka}
Stacionarita v širším smyslu se často používá v inženýrských aplikacích (závisí pouze na první dvou momentech).
\end{note}

Pokud je náhodný proces stacionární, pak nám to implikuje, že náhodný proces je stacionární v širším smyslu. Obrácená implikace obecně neplatí (pouze pro některé speciální náhodné procesy

\section{Vlastnosti autokorelační funkce}
Vlastnosti autokorelační funkce pro náhodné procesy stacionární v širším smyslu platí

\begin{itemize}
\item Autokorelační funkce je sudá
\[ R_{xx}(\tau)=R_{xx}(-\tau) \]

\item Pro $\tau=0$ je autokorelační funkce kladná
\[ R_{xx}(0) = \mean{X^2(t)}>0 \]

\item Autokorelační funkce dosahuje maxima pro $\tau=0$

\[ |R_{xx}(\tau)|\leq R_{xx}(0) \]
\end{itemize}

\begin{figure}
\caption{Autokorelační funkce}
\end{figure}

\section{Proces s nezávislými přírůstky}
Nechť $X(t),t\in T$ je náhodný proces. Protože v jakémkoliv okamžiku $t$ je $X(t)$ náhodná veličina, je přírůstek (rozdíl) $X(t_2)-X(t_1)$ pro dané $t_1$ a $t_2$ také náhodná veličina. Hustotu pravděpodobnosti přírůstku lze vypočítat z hustot pravděpodobnosti $X(t_1)$ a $X(t_2)$.

\subsection{Proces s nezávislými přírůstky (definice)}
Náhodný proces $X(t),t\in T$ má nezávislé přírůstky, pokud pro jakékoliv $t_1<t_2<t_3<t_4$.

\[ P_{x(t_2)-x(t_1),x(t_4)-x(t_3)}(x_2-x_1,x_4-x_3)=P_{x(t_2)-x(t_1)}(x_2-x_1)\cdot P_{x(t_4)-x(t_3)}(x_4-x_3) \]

Platí též

\[ P_{x(t_2)-x(t_1),x(t_3)-x(t_2)}(x_2-x_1,x_3-x_2) = P_{x(t_2)-x(t_1)}(x_2-x_1)\cdot P_{x(t_3)-x(t_2)}(x_3-x_2) \]

\textbf{Důležité:} Pokud $X(t),t\in T$ je proces s nezávislými přírůstky, platí, že je zároveň Markovův proces.

\subsection{Kategorie Markových procesů}

\begin{table}[H]
\centering
\begin{tabular}{r|cc}
& spočitatelný počest stavů & spočitatelná realizace\\\hline
diskrétní v čase & Markovské řetězce & -\\
spojité v čase & Markovské řetězce spojité v čase (Poissonův proces) & difúzní procesy (Wienerův proces)
\end{tabular}
\end{table}

\subsection{Věta Chapman-Kolmogorova}
Pro Markovův náhodný proces $X(t),t\in T$ platí

\[ P_{x(t_3)|x(t_1)}(x(t_3)|x(t_1))=\int\limits_{-\infty}^\infty P_{x(t_3)|x(t_1)}(x(t_3)|x(t_2))\cdot P_{x(t_2)|x(t_1)}(x(t_2)|x(t_1))\d x(t_2) \]

pro $t_1<t_2<t_3$, $p(x|y, z)p(y|z)=p(x,y|z)$.

\section{Markovské řetězce}
\begin{itemize}[noitemsep]
\item Množina $T$ je nejvýše spočetná $T=\brs{0,1,\ldots}$
\item Množina stavů je nejvýše spočetná $s=\brs{1,2,3,\ldots}$
\end{itemize}

\subsection{První hustota pravděpodobnosti}

\[ P(X(n)=s)=\Pi_s(n), \]

což představuje pravděpodobnost, že v čase $n$ se řetězec nachází ve stavu $s$. Lze uspořádat do vektoru

\[ \Pi^T(n) = [\Pi_1(n),\Pi_2(n),\ldots,\Pi_s(n),\ldots] \]

Počáteční rozložení pravděpodobnosti je $\Pi^T(0)$. První hustota pravděpodobnosti jednoznačně určuje první distribuční funkci

\[ F(x,n)=\sum_{s\leq x} \Pi_s(n)=P(X(n)\leq x) \]

\subsection{Druhá hustota pravděpodobnosti}
\begin{eqnarray*}
P(X(n)=s\wedge X(n+1)=r) & = & P(X(n+1)=r|X(n)=s)\cdot P(X(n)=s) =\\
& = & P_{sr}(n)\cdot\Pi_s(n)
\end{eqnarray*}

Výraz $P_{sr}$ je pravděpodobnost, že proces v čase $n$ přešel ze stavu $s$ do stavu $r$.

\begin{eqnarray*}
\Pi_r(n+1) & = & \sum_{s=1}^\infty P(X(n)=s\wedge X(n+1)=r) =\\
& = & \sum_{s=1}^\infty P_{sr}(n)\Pi_s(n)
\end{eqnarray*}

To lze maticově zapsat

\[ \vec{P}(n) =
\begin{bmatrix}
P_{11}(n) & P_{12}(n) & \cdots\\
P_{21}(n) & P_{22}(n) & \cdots\\
\vdots & \vdots & \ddots
\end{bmatrix} \]

$\Pi^T(n+1)= \Pi^T(n)\cdot P(n)$. Podobně platí 

\[ \Pi^T(n+2)=\Pi^T\cdot \underbrace{P(n)\cdot P(n+1)}_{\vec{P}^{(2)}(n)}, \]

kde $\vec{P}^{(2)}(n)$ představuje matici přechodu o 2 kroky. Obecně tedy platí

\[ \Pi^T(n+k)=\Pi^T(n)\cdot P(n)P(n+1)\ldots P(n+k-1) \]

Z věty o úplné pravděpodobnosti pro pravděpodobnost přechodu o 2 kroky plyne

\[ P_{ik}^{(2)}(n) = \sum_{s=1}^\infty P_{is}(n)P_{sk}(n+1) \]

(Počet stavů je početný, proto je ve vzorci použita sum a nikoliv integrál). Maticově lze zapsat jako $\vec{P}^{(2)}=P(n)\cdot P(n+1)$. Přechod o 3 kroky je vyjádřen

\[ P_{ik}^{(3)} = \sum_{s=1}^\infty \sum_{r=1}^\infty P_{is}(n)P_{sr}(n+1)P_{rk}(n+2) \]

\[ P^{(3)}(n)=P(n)P(n+1)P(n+2) \]

Obecně pro přechod o $k$ kroků

\[ P^{(k)}(n)=P(n)P(n+1)\cdots P(n+k-1) \]

\subsection{Vlastnosti Markovových řetězců}
Markovův řetězec může být

\begin{enumerate}[noitemsep]
\item \textbf{homogenní} -- pokud je matice přechodu $\vec{P}$ konstantní
\item \textbf{nehomogenní} -- pokud je $P(n)$ funkcí času
\end{enumerate}

Pro homogenní Markovův řetězec platí $P^{(k)}(n)=p^k$ a tedy

\[ \Pi\t(n+k) = \Pi\t(n)p^k \]
\[ m(n) = \mean{X(m)}=\sum_{s=1}^\infty s\Pi_s(n) \]

Autokovarianční funkce Markovova řetězce je

\[ C(n,q)=\mean{X(n),X(q)}=m(n)m(q)=\sum_{s=1}^\infty\sum_{k=1}^\infty s\cdot r\cdot P_{sr}^{q-n}(n)\Pi_s(n)-m(n)m(q), \]

kde $P_{sr}^{q-n}(n)$ je pravděpodobnost přechodu ze stavu $s$ v čase $n$ do stavu $r$ v čase $q$. Počáteční podmínky jsou

\[ P_{ss}^{(0)}=1, P_{sr}^{(0)}=0\quad\text{pro } s\neq r \]

\subsection{Stacionární Markovské řetězce}
\begin{itemize}
\item $\Pi\t(n)=\Pi\t(n+k)$, tzn. $\Pi\t(n)$ nezávisí na čase $n$, tedy $\Pi\t(n)=\Pi\t$.
\item $P(X(n)=s \wedge X(n+1)=r)=P(X(n+q)=s \wedge X(n+1+q)=r)=P_{sr}^{(q-n)}(n)\Pi_s=P_{sr}^{(\tau)}\Pi_s$, kde $\tau = q-n$ a značí, o kolik se posuneme nezávisle na čase (předchozí rovnost $P_{sr}^{(q-n)}(n)\Pi_s$ značí, o kolik se posuneme v čase $n$).
\item $m=\displaystyle\sum_{s=1}^\infty s\cdot\Pi_s=\text{konst.}$, což značí, že střední hodnota je konstanta
\item $c(\tau)=\sum_{s=1}^\infty\sum_{r=1}^\infty s\cdot r\cdot P_{sr}^{(\tau)}\Pi_s-m^2$, což značí, že autokovarianční funkce záleží na vzdálenosti dvou časových okamžiků 
\end{itemize}

\subsection{Regulární Markovské řetězce}
Markovův řetězec s maticí přechodu $\vec{P}$ je regulární, jestliže existuje takové $n$, že $P^n$ nemá nulové prvky. Pro regulární Markovův řetězec platí:

\begin{enumerate}
\item Vždy existují finální pravděpodobnosti $\Pi\t=[\Pi_1,\Pi_2,\ldots,\Pi_s,\ldots]$, pro něž platí

\[ \Pi_j=\lim_{n\to\infty}P_{ij}^{(n)}, \]

tedy že pravděpodobnosti nezávisí na počátečním stavu

\[ \lim_{n\to\infty}P^n = 
\begin{bmatrix}
\Pi_1 & \Pi_2 & \Pi_3 & \cdots \\
\Pi_1 & \Pi_2 & \Pi_3 & \cdots \\
\vdots & \vdots & \vdots & \ddots \\
\end{bmatrix} 
\]

\item Pro všechny pravděpodobnosti $\Pi_j$ platí $\Pi_j>0$

\item Vektor $\Pi\t$ je jediný, pro který platí

\[ \Pi\t=\Pi\t\cdot P \]
\end{enumerate}

\subsubsection{Věta}
U regulárního Markovského řetězce je střední hodnota počtu přechodů přes daný stav neomezená.\br

Důležitá úloha: Střední počet kroků prvního přechodu ze stavu $i$ do stavu $j$ (označíme $m_{ij}$). Předpokládejme, že se během prvního kroku dostaneme ze stavu $i$ do stavu $k$. Mohou nastat dvě situace:

\begin{enumerate}
\item Pokud $k=j$, pak $m_{ij}=1$.
\item Pokud $k\neq j$, pak na přechod do stavu $j$ je potřeba s pravděpodobností $P_{ik}$ potřeba $m_{ij}$ kroků, tedy

\[ m_{ij}=\sum_{\substack{k=1\\k\neq j}}^\infty P_{ik}m_{kj}+1,\quad i,j=1,2,\ldots \]

Toto lze zapsat v maticovém tvaru
\[ \vec{M} = \vec{P}\vec{\widehat{M}} + \vec{E}, \]

kde $\vec{E}$ je matice samých jedniček a $\vec{\widehat{M}}$ je jako matice $\vec{M}$, jen má na diagonále samé nuly.
\end{enumerate}

\subsubsection{Střední doba návratu do daného stavu}
Střední dobu návratu do stavu $i$ (označeno jako $m_{ii}$) vypočteme

\begin{eqnarray*}
\vec{M} & = & \vec{P}\cdot\vec{\widehat{M}}+\vec{E}\\
\vec{P}^n\vec{M} &=& \vec{P}^{n+1}\vec{\widehat{M}}+\vec{P}^n\cdot\vec{E}\\
\vec{P}^n\vec{M} &=& \vec{P}^{n+1}\vec{\widehat{M}}+\vec{E}\hspace*{2cm} \text{protože }\vec{P}\cdot\vec{E}=\vec{E}\\
\lim_{n\to\infty}\vec{\Pi}\vec{M} &=& \vec{\Pi}\vec{\widehat{M}}+\vec{E}\hspace*{2.55cm} \text{protože } \lim_{n\to\infty}\vec{P}^n=\vec{\Pi}\\
\vec{\Pi}(\vec{M}-\vec{\widehat{M}}) & = & \vec{E}
\end{eqnarray*}

Pro výraz $\vec{M}-\vec{\widehat{M}}$ platí

\[ \vec{M}-\vec{\widehat{M}} = 
\begin{cases}
m_{ij}=0 & \text{pokud } j\neq i\\
m_{ij}\neq 0 & \text{pokud } i = j
\end{cases}
 \]
 
Platí tedy, že $\Pi_i\cdot m_{ii}=1$, tedy $m_{ii}=\frac{1}{\Pi_i}$. Toto tvrzení vyplývá z

\[
\begin{bmatrix}
\Pi_1 & \Pi_2 & \Pi_3 & \cdots \\
\Pi_1 & \Pi_2 & \Pi_3 & \cdots \\
\vdots & \vdots & \vdots & \ddots
\end{bmatrix} \cdot
\begin{bmatrix}
m_{11} & \emptyset & \cdots \\
\emptyset & m_{22} & \cdots \\
\vdots & \vdots & \ddots
\end{bmatrix} = 
\begin{bmatrix}
1 & 1 & 1 & \cdots \\
1 & 1 & 1 & \cdots \\
\vdots & \vdots & \vdots & \ddots
\end{bmatrix}
\]