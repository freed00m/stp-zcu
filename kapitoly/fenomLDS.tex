\chapter{Průchod signálu LDS ve fenomenologickém popisu}
Budeme charakterizovat vlastnosti náhodných signálů po průchodu lineárním dynamickým systémem (LDS) definovaným vnějším popisem. Pro jednoduchost budeme uvažovat jednorozměrný systém (kauzální) popsaný impulsní charakteristikou $g_k$($g_k=0$ pro $k<0$). Vstupem systému bude náhodný proces (posloupnost) $u_k$ s konečnými druhými momenty. Střední hodnotu označíme $\mean{u_k}=m_k^u$ s autokovarianční funkcí

\[ C_{uu}(k,e) = \mean{(u_k-m_k^u)(u_e-m_e^u)} \]

pro výstup platí

\[ y_k = \sum_{l \to \infty}^k g_{k-l}u_e=\sum_{l=0}^{\infty} g_l u _{k-l} \]

Pro stabilní systém řada konverguje (\uv{ve středně-kvadratickém smyslu}) a výstup $y_k$ je zase náhodná posloupnost s konečými druhým momenty

\[ m_k^y = \mean{y_k}=\mean{\sum_{l=0}^\infty g_l u_{k-l} } = \sum_{l=0}^\infty g_lm_{k-l}^u \]

Pro výpočet autokovarianční funkce si definujeme odchylky $\widetilde{u}^k=u_k-m_k^u$ a $\widetilde{y}_k=y_k-m_k^u$, pro které platí

\[
\widetilde{y}_k = y_k-m_k^y = \sum_{l=0}^\infty g_l u_{k-l}-\sum_{l=0}^\infty g_l m_{k-l}^u=\sum_{l=0}^\infty g_l \widetilde{u}_{k-l}
\]

Rozdíl mezi signálem a jeho střední hodnotou se filtruje stejně, jako střední hodnota. Autokovarianční funkce výstupu pak bude

\begin{eqnarray*}
C_{yy}(k,l) &=& \mean{\widetilde{y}_k\, \widetilde{y}_l}=\mean{ \sum_{i=0}^\infty g_i \widetilde{u}_{k-i}\cdot \sum_{j=0}^\infty g_j\widetilde{u}_{l-j} } = \\
& = & \sum_{i=0}^\infty \sum_{j=0}^\infty g_i g_j\mean{\widetilde{u}_{k-i}\, \widetilde{u}_{k-j}} = \\
& = & \sum_{i=0}^\infty \sum_{j=0}^\infty g_i g_j C_{uu}(k-i,l-j)
\end{eqnarray*}

Podobně vzájemná autokovarianční funkce mezi výstupem $y_k$ a vstupem $u_k$.

\begin{eqnarray*}
C_{yu}(k,l) &=& \mean{\widetilde{y}_k\, \widetilde{u}_l}=\mean{ \sum_{i=0}^\infty g_i \widetilde{u}_{k-i}\cdot \widetilde{u}_l } = \\
& = & \sum_{i=0}^\infty g_i\mean{\widetilde{u}_{k-i}\, \widetilde{u}_{l}} = \\
& = & \sum_{i=0}^\infty g_i  C_{uu}(k-i,l)
\end{eqnarray*}

Je-li vstupní proces gaussovský, je i výstupní proces gaussovský (jenom tranformujeme náhodné veličiny, vynásobíme konstantou a sečteme, proto zůstane g. proces gausovský). Zjednodušení výsledků, pokud bude vstupní proces stacionární, tedy


Autokovarianční funkce výstupu

\[ C_{yy}(k,e) = \sum_{i=0}^\infty \sum_{j=0}^\infty g_i g_j C_{uu}(k-e+j-i)=C_{yy}(k-e) \]

Vzájemná autokovarianční funkce výstupu a vstupu

\[ C_{yu}(k,e) = \sum_{i=0}^\infty g_i C_{uu}(k-e-j)=C_{yu}(k-e) \]

Střední hodnota výstupu je konstantní (nemění se v čase), obě autokovarianční funkce jsou funkcí rozdílu časových okamžiků. Výstupní proces je tedy stacionární v širším smyslu a výstup a vstup jsou vzájemně stacionární.

\[ m_k^u = m^u, C_{uu}(k,l)=C_{uu}(k-l) \]

 Lze psát, rovnice pro střední hodnotu

kde
\[ K=\sum_{i=0}^\infty g_i \]

střední hodnota

\[ m_k^y\]


\[ C_{yy}(k,l) = \sum_{i=0}^\infty \sum_{j=0}^\infty g_i g_j C_{uu}(k-i-j)= C_{yy}, \]

vzájemná kovarianční funkce

\[ C_{yu}(k,l) = \sum_{i=-0}^\infty g_i C_{uu}(k-i-l)=C_{uu}(k-l), \]


střední hodnota výstupu je konstantní, obě autokovarianční fce jsou funkcemi č. okamžiků, tj. výstup je staionární v šírším smyslu a vstup a výstup jsou vzájemně stacionární



někdy se zapisují pomocí konvolutorního operátoru $*$ je, vzájemná autokovarianční funkce je konvolucí impulsní funkce a autokovarianční funkce vstupu.

\[ C_{yu} (k)  =\sum_{i=-\infty}^\infty g_i C_{uu} (k-i) = g * C_{uu} \]

vzájemná kovarianční funkce výstupu a vstupu je konvolucí impulsní funkce a autokovarianční funkce vstupu.

Obdobně to můžeme udělat pro autokovarianční funkci výstupu

\[ C_{yy} (K) = \sum_{i=-\infty}^\infty \sum_{j=-\infty}^\infty g_i g_j C_{uu} (k+j-i) = \sum_{i=-\infty}^\infty \sum_{j=-\infty}^\infty g_i C_{uu}(k-j-i) g_{-j} = g * C_uu * \bar{g}   \]

kde $\bar{g}_k=g_{-k}$,

\uv{To je jen šikovný zápis :)}