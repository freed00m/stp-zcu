\chapter{Střední hodnota a transformace náhodných veličin}

\section{Střední hodnota}
Náhodná veličina je plně charakterizována svojí distribuční funkcí nebo hustotou pravděpodobnosti. Mnohdy vystačíme s neúplným popisem náhodné veličiny, kdy na úkor úplnosti můžeme získat jednoduchý tvar stochastické charakteristiky náhodné veličiny. Takovou jednoduchou charakteristikou je \textbf{střední hodnota} náhodné veličiny.

\subsection{Střední hodnota diskrétní náhodné veličiny}
Mějme náhodnou proměnnou $X$, která na množině $\mathbb{R}$ nabývá konečného, ale nejvýše spočetného, počtu hodnot se nazývá diskrétní náhodná veličina. Budeme opakovat náhodný pokus n-krát a budou nás zajímat průměrné hodnoty, kolem níž se budou výsledky soustřeďovat n-náhodných pokusů v $n_1$ případech bude realizace $X$ rovna $x_1$, $\ldots$, v $n_k$ případech bude realizace $X$ rovna $x_k$, pak aritmetický průměr je

\[ \bar{x}=\frac{1}{n}\sum_{i=1}^k n_ix_i, \]

kde $n_i\approx P(X(\omega)=x_i)\cdot n$. Dosadíme $n_i$ do rovnice a dostáváme

\[ \bar{x}\approx \sum_{i=1}^k P(X(\omega)=x_i)\cdot x_i, \]

označíme $P(x_i)=P(X(\omega)=x_i)$. Střední hodnota náhodné veličiny se značí $\mean{X}$ a je rovna

\[ \mean{X} = \sum_{i=1}^k P(x_i)\cdot x_i \]

\subsection{Střední hodnota spojité náhodné veličiny}
Náhodná veličina $X$ nabývá na množině $\mathbb{R}$ nespočetného množství hodnot a nazývá se spojitou náhodnou veličinou. Střední hodnota této náhodné veličiny je (analogicky vztah diskrétní střední hodnoty s integrálem místo sumy)

\[ \mean{X} = \int\limits_{-\infty}^\infty xp(x)\d x, \]

kde $p(x)$ je hustota pravděpodobnosti náhodné veličiny $X$.

\subsection{Střední hodnota vektorové náhodné veličiny}

\[ \mean{\vecrow[x]}=\int\limits_{-\infty}^\infty\cdots\int\limits_{-\infty}^\infty(\vecrow[x])p(\vecrow[x])\vecrow[\d x] \]
nebo
\[\mean{\vec{x}}=\int\limits_{-\infty}^\infty \vec{x}p(\vec{x})\d\vec{x}  \]

\subsection{Střední hodnota funkce náhodné veličiny}
Nechť je dán prostor $\brs{\Omega, \mathscr{F}, P}$, náhodnou veličinu $X:\Omega\to\mathbb{R}$, kde $\mathbb{R}$ je množina reálných čísel s přirozenou topologií. Nechť je dána měřitelná funkce $\varphi:\mathbb{R}\to\mathbb{R}$, která každé hodnotě $X(\omega)$ náhodné veličiny $X$ přiřazuje hodnotu $Y(\omega)=\varphi[X(\omega)],Y(\omega)\in\mathbb{R}$. Zobrazení je tedy také náhodná veličina a platí

\[ E[Y] = E[\varphi(X)]=\sum_{i=1}^k \varphi(x_i)p(x_i) \]

pro diskrétní náhodnou veličinu. Pro spojitou náhodnou veličinu platí analogie s integrálem

\[ E[Y] = E[\varphi(X)]=\int\limits_{-\infty}^\infty \varphi(x)p(x_i) \]

\subsection{Momentové charakteristiky náhodných veličin}
Střední hodnota nepopisuje náhodnou veličinu úplně. Bude-li dán soubor funkcí $\varphi_i$ pro $i=1,2,\ldots$ náhodné veličiny $X$, pak znalost středních hodnot již může být úplným popisem náhodné veličiny $X$. Je-li takovým souborem funkcí 

\[ \varphi_i(X) = X^i, i=1,2,\ldots \]

pak střední hodnoty $E[\varphi_i(X)]=E[X^i]$ jsou obecnými momenty náhodné veličiny $X$. \textbf{Významnou roli} mají momenty prvního a druhého řádu. Momentem prvního řádu je \textbf{střední hodnota}, momentem druhého řádu se bere centrovaný moment a pro spojitou náhodnou veličinu má tvar

\[ E[(X-E[X])(X-E[X])^T] = \int\limits_{-\infty}^\infty (X-E[X])(X-E[X])^T p(x)\d x \]

Jedná se o kovarianční matici náhodného vektoru $\vec{x}$. Její prvek na i-tém řádku a j-tém sloupci je dán

\[ E[(X_i-E[X_i])(X_j-E[X_j])^T]= \int\limits_{-\infty}^{\infty}\int\limits_{-\infty}^{\infty} (X_i-E[X_i])(X_j-E[X_j])^T p(x_i)p(x_j)\d x_i\d x_j\]

a nazýváme jej (vzájemnou) kovariancí náhodných veličin $x_i$ a $x_j$. Pro $i=j$ se tento prvek nazývá variancí, nebo též rozptylem, náhodné veličiny $X_i$. Odmocninou rozptylu je směrodatná odchylka.

\begin{note}{Poznámka}
První centrální moment je nulový, protože pro $E[(X-E[X])^1]=E[X]-E[X]^1$ (lze vytknout, neboť střední hodnota je lineární operátor).
\end{note}

\subsection{Podmíněné střední hodnoty}
Dvě náhodné veličiny $X$ a $Y$ na $\brs{\Omega,\mathscr{F},P}$ a nechť je dána podmíněná pravděpodobnost

\[ P(X(\omega)=x_i | Y(\omega)=y_j)=P(x_i|y_j) \]

pro diskrétní náhodné veličiny a podmíněná hustota pravděpodobnosti $p(x|y)$ pro spojité náhodné veličiny.

\[ E[\varphi(X)|Y(\omega)=y]=\sum_{i=1}^k \varphi(x_i)p(x_i|y) \]

pro diskrétní náhodné veličiny. Pro spojité náhodné veličiny analogicky platí

\[ E[\varphi(X)|Y(\omega)=y]=\int\limits_{-\infty}^\infty \varphi(x)p(x|y)\d x \]

\begin{note}{Poznámka}
Podmíněná střední hodnota je funkcí náhodné veličiny $Y$ a je tedy sama náhodnou veličinou.
\end{note}

\subsection{Vlastnosti střední hodnoty}
\subsubsection{Linearita}
\[ E[c_1X_1 + c_2X_2] = c_1E[X_1] + c_2E[X_2] \]

v případě, že $c_1$ a $c_2$ jsou konstanty.

\subsubsection{Lineární pravděpodobnost}
\[ E[X] = E[E[X|Y(\omega)=y]] \]

\subsubsection{Nemultiplikativnost}
\[ E[X\cdot Y] = E[X]\cdot E[Y], \]

pokud jsou $X$ a $Y$ nezávislé.

\[ E[(X-E[X])(X-E[X])^T] = E[XX^T]-E[X]E[X]^T \]

\section{Transformace náhodných veličin}
Uvažujme spojitou náhodnou veličinu $X$, spojitou distribuční funkci $F(x)$ a hustotu pravděpodobnosti $p_X(x)$. Dále máme funkci $\varphi$, která zobrazuje náhodnou veličinu $X$ na náhodnou veličinu $Y$, tzn. $Y=\varphi(X)$. Cílem je najít distribuční funkci $F_Y(y)$, případně $p_Y(y)$.\br

\textbf{A)} Předpokládejme, že $\varphi$ je monotónní, rostoucí a diferencovatelná

\begin{figure}
	\begin{tikzpicture}[trim axis left, trim axis right]
		\begin{axis}[
			height=6cm,
			width = 7.5cm,
			axis lines=middle,
			xlabel={$x$},
			ylabel={$y$},
			ytick = {},
			xtick = {}
			%grid = both
			]
			\addplot [red, domain=1:10] {
				atan(x-5)+90
			};
		\end{axis}
	\end{tikzpicture}
\end{figure}

Distribuční funkce $F_Y(y)=P(Y<y)$ je pravděpodobnost, že $Y(\omega)<y$, a to nastává tehdy, jestliže $X(\omega)$, kde $x=\varphi^{-1}(y),\varphi^{-1}$ je inverzní funkce k $\varphi$. Platí tedy

\[ F_Y(y)=P(Y<y)=P(X<\varphi^{-1}(y)) \]

nebo-li $F_Y(y)=F_X(\varphi^{-1}(y))$, hustota pravděpodobnosti $p_Y(y)$ je daná

\[ p_Y(y) = \frac{\d F_Y(y)}{\d y}=\frac{\d F_x}{\d x}(\varphi^{-1}(y))\cdot\frac{\d\varphi^{-1}(y)}{\d y}, \]

kde

\[ \frac{\d F_x}{\d x}(\varphi^{-1}(y))=p_X(\varphi^{-1}(y)) \]
a

\[ P_Y(y)=p_x(\varphi^{-1}(y))\frac{\d\varphi^{-1}(y)}{\d y} \]

\textbf{B)} Funkce $\varphi$ je monotónní, ale klesající.

\begin{figure}
\begin{figure}
	\begin{tikzpicture}[trim axis left, trim axis right]
		\begin{axis}[
			height=6cm,
			width = 7.5cm,
			axis lines=middle,
			xlabel={$x$},
			ylabel={$y$},
			ytick = {},
			xtick = {}
			%grid = both
			]
			\addplot [red, domain=1:10] {
				-atan(x-5)+90
			};
		\end{axis}
	\end{tikzpicture}
\end{figure}
\end{figure}

\begin{eqnarray*}
P(X<x)&=&1-P(X\geq x)\\
\underbrace{P(X<x)+P(X\geq x)}_{P(X\in\mathbb{R})} & = & 1
\end{eqnarray*}

Protože $F_Y(y)=P(Y<y)=P(X\geq x)=1-P(X<x)$, pro $x$ platí $X=\varphi^{-1}(y)$ a platí $F_Y(y)=1-F_X(\varphi^{-1}(y))$. Pro derivaci platí

\[ p_Y(y) = \frac{\d F_Y(y)}{\d y}= -\frac{\d F_X}{\d x}(\varphi^{-1}(y))\cdot\frac{\d\varphi^{-1}(y)}{\d y} \]

Funkce $\varphi$ je klesající a derivace je kladná. Vztahy pro hustotu pravděpodobnosti můžeme spojit

\[ p_Y(y)=p_X(\varphi^{-1}(y))\left|\frac{\d\varphi^{-1}(y)}{\d y}\right| \]

To platí bez ohledu na to, zda je funkce $\varphi$ klesající nebo rostoucí.

\subsubsection{Alternativní podoba}
\[ y=\varphi(\varphi^{-1}(y)) \]

Derivací obou stran dle $y$ dostáváme
\[ 1=\frac{\d\varphi}{\d x}\cdot\frac{\d\varphi^{-1}}{\d y}\Rightarrow\frac{\d\varphi^{-1}}{\d y}=\frac{1}{\frac{\d\varphi}{\d x}} \]

\[ p_Y(y) = \frac{p_X(\varphi^{-1}(y)}{\left|\frac{\d\varphi}{\d x}(\varphi^{-1}(y))\right|} \]

Uvedené vztahy mají smysl, jestliže mají smysl všechny použité operace. K tomu stačí, aby $F_X(x)$ byla všude spojitá a aby (s výjimkou nanejvýše konečného počtu bodů) existovala $p_X(x)$. Dále musí platit, že $\varphi(x)$ je ryze monotónní a všude diferencovatelná.

\subsection{Zobecnění pro vektorový případ}
Mějme náhodný vektor $\vec{x}=[\vecrow[x]]^T$ a hustotu pravděpodobnosti\\ $p_{\vec{x}}(\vec{x})=p_{\svecrow[x]}(\vecrow[x])$. Je-li $\varphi$ prosté zobrazení z $\mathbb{R}^n$ do $\mathbb{R}^n$, pak náhodný vektor $\vec{Y}=[\vecrow[Y])^T$ daný vztahem $\vec{Y}=\varphi(\vec{X})$ má hustotu pravděpodobnosti

\[ p_{\vec{Y}}(\vec{y}=p_{\vec{x}}(\varphi^{-1}(\vec{y})\cdot\left|\det\frac{\d\varphi^{-1}(\vec{y})}{\d\vec{y}}\right| \]

Po úpravě získáváme

\[ p_{\vec{Y}}(\vec{y})=\frac{p_{\vec{X}}(\varphi^{-1}(y)}{\left|\det\frac{\d\varphi}{\d x}(\varphi^{-1}(y))\right|}, \]

kde $\varphi_1=\varphi_1(\vecrow[x])$ až $\varphi_n=\varphi_n(\vecrow[x])$ jsou prvky Jacobiho matice

\[ \frac{\d\varphi}{\d x}=\begin{bmatrix}
\frac{\d\varphi_1}{\d x_1} &  \cdots & \frac{\d\varphi_1}{\d x_n}\\
\vdots & & \vdots\\
\frac{\d\varphi_n}{\d x_1} & \cdots & \frac{\d\varphi_1}{\d x_n}
\end{bmatrix} \]