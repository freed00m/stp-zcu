\chapter{Spektrální faktorizace diskrétního náhodného procesu}
Stejně jako u lineárních t-invariantních deterministických systémů lze vztahy mezi vstupy a výstupy popisovat ve frekvenční oblasti. Označíme

\[ G(z) = \sum_{i=0}^\infty g_iz^{-i} \]

Tato rovnice vyjadřuje přenos v Z-transformaci. Definujeme spektrální hustotu procesu jako Z-transformaci autokovarianční (popřípadě vzájemné kovarianční) funkce vyčíslené v jednotkové kružnici.

\[ z=e^{j\omega T_s},\quad \omega\in(-\omega_n,\omega_n), \]

kde $\omega_n$ je tzv. Nyquistova frekvence $\omega_n = \frac{\pi}{T_s}$. Označme

\begin{alignat*}{2}
S_{uu}(\omega) & = S_{uu}(z)|_{z=e^{j\omega T_s}}, S_{uu}(z) && = \sum_{i=-\infty}^\infty C_{uu}(i)z^{-i}\\
S_{yy}(\omega) & = S_{yy}(z)|_{z=e^{j\omega T_s}}, S_{yy}(z) && = \sum_{i=-\infty}^\infty C_{yy}(i)z^{-i}\\
S_{yu}(\omega) & = S_{yu}(z)|_{z=e^{j\omega T_s}}, S_{yu}(z) && = \sum_{i=-\infty}^\infty C_{yu}(i)z^{-i}
\end{alignat*}

Tato definice spektrální hustoty odpovídá formálně Fourierově transformaci (auto) kovarianční funkce. Platí:

\begin{eqnarray*}
S_{yu}(z) & = & \sum_{i=-\infty}^\infty C_{yu}(i)z^{-i} = \sum_{i=-\infty}^\infty z^{-i} \sum_{j=0}^\infty C_{uu}(i-j) = \sum_{i=-\infty}^\infty \sum_{j=0}^\infty z^{-i} g_j C_{uu}(i-j) = \\
& = & \sum_{i=-\infty}^\infty \sum_{j=0}^\infty z^{-j} g_j z^{-(i-j)} C_{uu}(i-j) = \sum_{j=0}^\infty z^{-j}g_j \sum_{i=-\infty}^\infty z^{-(i-j)} C_{uu}(i-j) =
\end{eqnarray*}

Odtud platí

\[ S_{yu}(z) = G(z)\cdot S_{uu}(z) \]

neboli

\[ S_{yu}(\omega) = G\left(e^{j\omega T_s}\right)\cdot S_{uu}(\omega) \]

Pro opačné pořadí vstupu a výstupu platí

\[ S^{uy}(z) = S_{uu}(z)\cdot G^*(z), \]

kde $G^*(z)=G(z)|_{z=z^{-1}}$, tedy

\[ S_{uy}(\omega) = S_{uu}(\omega)\cdot G^*\left(e^{j\omega T_S}\right)  \]

Pro spektrální hustotu výstupu platí

\[ S_{yy}(z) = \sum_{i=-\infty}^\infty z^{-i}C_{yy}(i) = \cdots = \sum_{j=0}^\infty z^{-j}g_j \sum_{i=-\infty}^\infty z^{-i} C_{uu}(i)\sum_{k=0}^\infty z^{-k}g_k \]

tedy

\begin{eqnarray*}
S_{yy}(z) & = & G(z)\cdot S_{uu}(z)\cdot G^*(z)\\
S_{yy}(\omega) & = & \left(G\left(e^{-j\omega T_s}\right)^2\right)\cdot S_{uu}(\omega)
\end{eqnarray*}

\subsection{Fyzikální interpretace}
Protože 

\[ \frac{1}{2\omega_n}\int\limits_{-\omega_n}^{\omega_n}S_{uu}(\omega)\d\omega = C_{uu}(0)=\delta_u^2=\mean{u_k^2} \]

Je měřítkem výkonu procesu, lze hodnotu (výkonové) spektrální hustoty v bodě $\omega$ chápat jako hustotu výkonu v okolí frekvence $\omega$. Součin

\[ G\left(e^{j\omega T_s}\right)\cdot G\left(e^{-j\omega T_s}\right) = \left|G\left(e^{j\omega T_s}\right)\right|^2 \]

udává výkonové zesílení na této frekvenci a hodnota $S_{yy}(\omega)$ je pak hustota výkonu výstupního signálu v okolí této frekvence.

\section{Využití vztahů pro vzájemnou autokovarianční funkci a vzájemnou spektrální hustotu pro identifikaci systému}

Uvažujme jako vstupní proces diskrétní bílý šum s autokovarianční funkcí

\[ C_{uu}(k) = \delta(k), \]

kde $\delta$ je Diracův impuls a spektrální hustotou

\[ S_{uu}(z) = \sum_{i=-\infty}^\infty C_{uu}(i)z^{-i} = \sum_{i=-\infty}^\infty \delta(i)z^{-i}=z^0=1 \]

Pro tento vstup získáme 

\begin{eqnarray*}
C_{yu}(k) & = & \sum_{i=-\infty}^\infty g_iC_{uu}(i-k) = \sum_{i=-\infty}^\infty g_i\delta(k-i)=g_k\\
S_{yu}(\omega) & = & G\left(e^{-j\omega T_s}\right)\cdot S_{uu}(\omega) = G\left(e^{j\omega T_s}\right)
\end{eqnarray*}

Vzájemná autokovarianční funkce je tedy rovna impulsní charakteristice systému a vzájemná spektrální hustota je rovna jeho frekvenčnímu přenosu.

\section{Spektrální faktorizace diskrétního náhodného procesu}

Kovarianční matice a spektrální události náhodného procesu závisí pouze na odchylkách uvažovaných signálů. Budeme uvažovat nulový deterministický vstup $u$ a vstupem budou pouze bílé šumy $w$ a $v$. Popis systému:

\begin{eqnarray*}
x_{k+1} & = & \vec{A}x_k + w_k\\
y_k & = & \vec{C}x_k + v_k
\end{eqnarray*}

pokud je $A$ stabilní, po uplynutí dostatečně dlouhé doby od $k=0$ budou stav i výstup stacionární náhodné procesy s nulovou střední hodnotou. K určení jejich kovarianční matice je nutno řešit Ljapunovovu rovnici a rovnici pro výstup.

\begin{eqnarray*}
P^x & = & \vec{A}P^x\vec{A}\t + \vec{Q}\\
P^y & = & \vec{C}P^x\vec{C}\t + \vec{R}
\end{eqnarray*}

Označme
\[ u_k = 
\begin{bmatrix}
w_k\\v_k
\end{bmatrix}
\]

Potom přenos mezi tímto vstupem a výstupem $y_k$ je

\[ G(z) = [C(zI-A)^{-1}, I] \]

Spektrální hustota vstupu je 

\[ S_{uu}(z) =
\begin{bmatrix}
Q & S\\
S\t & R
\end{bmatrix}
\]

a potom

\[
S_{yu}(z) = [C(zI-A)^{-1}, I]\cdot
\begin{bmatrix}
Q & S\\ S\t & R
\end{bmatrix}
\]

\[ S_{uy}(z) = S\t_{yu}(z) \]

\[ S_{yy}(z) = [C(zI-A)^{-1}, I]\cdot 
\begin{bmatrix}
Q & S\\ S\t & R
\end{bmatrix}\cdot [C(zI-A)^{-1}, I]\t
\]

Pro vzájemně nezávislé šumy $w_k$ a $v_k$ s kovarianční maticí

\[ \mathrm{cov}\left\{ \begin{bmatrix} w_k\\v_e \end{bmatrix}, \begin{bmatrix} w_e\\ v_e \end{bmatrix} \right\} = \mean{\begin{bmatrix} w_k\\v_e \end{bmatrix}\cdot \begin{bmatrix} w_e\\ v_e \end{bmatrix}\t} = \begin{bmatrix}
Q & 0 \\ 0 & R
\end{bmatrix}\delta(k-e) \]

Bude platit

\[ S_{yy}(z) = C(zI-A)^{-1}\cdot Q(zI-A)^{-\mathrm{T}}C\t + R \]

\begin{note}{Příklad rozkladu spektrální hustoty}
Uvažujme spektrální hustotu, která je nezápornou racionální funkcí $\cos(\omega T_s)$

\[ S_{yy}(z) = \frac{(z+0.5)(z^{-1}+0.5)}{(z+0.25)(z^{-1}+0.25)} \]

Pro $S_{uu}(\omega) = 1$ platí

\[ S_{yy}(z) = G(z)\cdot S_{uu}(z)\cdot G^*(z) = C(z)\cdot G^*(z),\quad G^*(z)=G(z)|_{z=z^{-1}} \]

Tuto racionální spektrální hustotu lze vyjádřit čtyřmi způsoby:

\begin{alignat*}{2}
G_1(z) & = \frac{z+0.5}{z+0.25} && \to \text{stabilní}\\
G_2(z) & = \frac{1+0.5z}{z+0.25} && \to \text{stabilní}\\
G_3(z) & = \frac{z+0.5}{1+0.25z} && \to \text{nestabilní}\\
G_4(z) & = \frac{1+0.5z}{1+0.25z} && \to \text{nestabilní}
\end{alignat*}

Stochastický signál s danými spektrálními vlastnostmi můžeme získat průchodem bílého šumu dynamickými systémy s přenosy $G_1$ a $G_2$ z nich $G_1$ má minimální fázi. Z toho plyne, že také inverze má stabilní přenos. Takovému rozkladu spektrální hustoty říkáme \textbf{spektrální faktorizace náhodného procesu}.
\end{note}

\subsection{Věta o spektrální faktorizaci}

Předpokládejme, že $S_{yy}(\omega)$ je racionální funkcí $\cos(\omega T_s)$ případně $e^{j\omega T_s}$, pak existuje právě jedna monická (= 1 u nejvyšších mocnin polynomů v čitateli i jmenovateli) racionální funkce $G(z)$, která má všechny póly uvnitř jednotkové kružnice a všechny nuly uvnitř, nebo na jednotkové kružnici. Platí

\[ S_{yy}(\omega) = G^2\left|G\left(e^{j\omega T_s}\right)\right|^2 \]

Uvažujme náhodný proces vzniklý průchodem bílého šumu s rozptylem $\sigma^2$ dynamickým systémem s přenosem $G(z)$ dle předchozí věty. Jeho spektrální hustota pak bude $S_{yy}(z)$.

\subsection{Věta o realizaci}
Pro každou racionální spektrální hustotu $S_{yy}\geq 0$ existuje právě jeden dynamický systém s přenosem $G(z)$, který je monickou racionální funkcí, má všechny póly uvnitř jednotkové kružnice a nuly uvnitř, nebo na jednotkové kružnici takový, že stacionární proces s danou spektrální hustotou můžeme získat jako výstup systému je-li vstupem diskrétní bílý šum.

\subsubsection{Důsledek}
Omezíme-li se na stacionární procesy s racionálními spektrálními hustotami, lze takové procesy získat filtrací diskrétního bílého šumu. Takové náhodné procesy lze reprezentovat ve tvaru:

\[ y_k = \frac{c(d)}{a(d)}e_k, \]

kde $e_k$ je bílý šum s rozptylem $\sigma^2$, $d$ je operátor zpoždění $d\cdot y_k=y_{k-1}$ a $c(d),a(d)$ jsou polynomy:

\begin{eqnarray*}
a(d) & = & 1 + a_1d + \cdots + a_{n_a}d^{n_a}\\
c(d) & = & 1 + c_1d + \cdots + a_{n_c}d^{n_c}
\end{eqnarray*}

nebo v časové oblasti

\[ y_k + a_1y_{k+1}+\cdots+a_{n_a}y_{k-n_a}=e_k+c_1e_{k-1}+\cdots+c_{n_c}e_{k-n_c} \]

Tato reprezentace se nazývá ARMA model (Autoregressive–moving-average model). Podle věty o realizaci lze nalézt polynomy $a(d)$ a $c(d)$ tak, aby byly oba stabilní (kromě kořenů $c(d)$ na jednotkové kružnici). Původní transformaci lze tedy invertovat:

\[ e_k=\frac{a(d)}{c(d)}y_k \]

Posloupnost $\{\ldots, y_{k-1}, y_k \}$ obsahuje veškeré informace o měřené posloupnosti $\{ \ldots, e_{k-1},e_k \}$. Ukážeme si důležitou vlastnost této bílé posloupnosti a zavedeme pojem \textbf{inovace}.

\subsection{Inovační reprezentace}
Uvažujme

\[ y_{k+1}=\frac{c(d)}{a(d)}e_{k-1} \]

a provedeme dělení polynomů

\[ \frac{c(d)}{a(d)} = 1+\frac{d\bar{c}(d)}{a(d)} \]

tedy

\[ y_{k+1} = \frac{\bar{c}(d)}{a(d)}\cdot e_k+e_{k+1} \]

Dosadíme za $e_k$ a dostaneme

\[ y_{k+1} = \frac{\bar{c}(d)}{a(d)}\cdot\frac{a(d)}{c(d)}y_k+e_{k+1}=\frac{\bar{c}(d)}{c(d)}y_k+e_{k+1} \]

Z toho plyne, že hodnotu $e_{k+1}$ můžeme interpretovat jako tu část informace o $y_{k+1}$, která není obsažena v minulé historii procesu $\{ \ldots,y_{k-1},y_k \}$, a proto ji nazýváme \textbf{inovace procesu}. Protože $e_k$ je bílá posloupnost, je $\widehat{y}_{k+1}=\frac{\bar{c}(d)}{c(d)}y_k$ nejlepší predikcí výstupu na základě zadané historie procesu $\{ \ldots, y_{k-1},y_k \}$ ve smyslu střední kvadratické chyby $\mean{(\ldots, y_{k+1}-\widehat{y}_{k+1})^2}$ a reprezentaci

\[ y_k = \frac{c(d)}{a(d)}e_k = \sum_{r=0}^\infty g_r e_{k-r} \]

nazýváme \textbf{inovační reprezentace}.

\begin{note}{Poznámka}
pracujeme-li s inovační reprezentací, tj. vstupní posloupnost je bílá, lze model převést do tvaru prediktoru. Jestliže však vstupní posloupnost není bílá, není $\widehat{y}_{k+1}$ dle předchozího vztahu optimální prediktor, neboť $e_{k+1}$ je korelována s předchozími daty, a proto může být chyba predikce dále zmenšena.
\end{note}

\begin{note}{Příklad - optimální prediktor výstupu}
Uvažujme spektrální hustotu

\[ S_{yy}(z) = \frac{(z+0.5)(z^{-1}+0.5)}{(z+0.25)(z^{-1}+0.25)} \]

Podle věty o realizaci lze tuto hustotu získat průchodem bílého šumu filtrem.

\[ y_k = \frac{1+0.5d}{1+0.25d}e_k,\quad \left(G(z) = \frac{z+0.5}{z+0.25}\right) \]

protože

\[ \frac{1+0.5d}{1+0.25d} = 1+\frac{0.25d}{1+0.25d} \]

lze proces popsat

\[ y_{k+1} = \frac{0.25d}{1+0.5d}y_k+e_{k+1} \]

a optimální prediktor výstupu $y_{k+1}$ je

\[ \widehat{y}_{k+1} = \frac{0.25d}{1+0.5d}y_k \] 

\end{note}